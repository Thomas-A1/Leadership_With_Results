\documentclass[12pt,a4paper]{article}
\usepackage[utf8]{inputenc}
\usepackage[T1]{fontenc}
\usepackage{geometry}
\usepackage{graphicx}
\usepackage{hyperref}
\usepackage{amsmath}
\usepackage{booktabs}
\usepackage{array}
\usepackage{enumitem}
\usepackage{fancyhdr}
\usepackage{setspace}
\usepackage{url}

% Page setup
\geometry{margin=1in}
\onehalfspacing
\emergencystretch=3em
\sloppy

% Header and footer
\pagestyle{fancy}
\fancyhf{}
\fancyhead[L]{MSc Intelligent Computing Systems}
\fancyhead[R]{Leadership Seminar}
\fancyfoot[C]{\thepage}
\renewcommand{\headrulewidth}{0.4pt}

% Section formatting - using standard LaTeX commands

% Hyperlink setup
\hypersetup{
    colorlinks=true,
    linkcolor=blue,
    filecolor=magenta,      
    urlcolor=cyan,
    citecolor=blue
}

% Title information
\title{\textbf{Leadership Analysis Report: Sankofa Digital Solutions}\\
\large PROSIT on Goleman's Leadership That Gets Results (2000)}
\author{Team 2\\
Thomas Kojo Quarshie\\
Betty Odaako Blankson\\
Chelsea Maame Abena Pokuaah Owusu\\
Innocent Farai Chikwanda\\
Nana Sam Yeboah}
\date{November 2024}

\begin{document}

\maketitle

\newpage

\tableofcontents
\newpage

\section{Executive Summary}

\noindent This report presents a comprehensive analysis of leadership styles and organizational climate at Sankofa Digital Solutions (SDS), a prominent Ghanaian technology company specializing in digital government platforms and AI-enabled data analytics. The analysis is based on Goleman's (2000) framework of leadership styles and organizational climate drivers. Our investigation reveals that SDS faces significant challenges stemming from leadership behaviors rather than technical deficiencies. The delayed release of Sankofa Insights 4.0, deteriorating staff morale, and coordination problems are directly linked to the inappropriate application of leadership styles and deficits in emotional intelligence competencies among the executive team.

\noindent The report diagnoses four dominant leadership styles currently in use, identifies critical missing styles, analyzes their impact on organizational climate, and proposes a comprehensive 90-day intervention plan. A detailed presentation of our findings is available online at: \url{https://raw.githack.com/Thomas-A1/Leadership\_With\_Results/main/Leadership\_Presentation.html}

\section{Introduction}

\noindent Sankofa Digital Solutions has established itself as a leader in digital government platforms and AI analytics across West Africa over the past decade. However, the company currently faces operational, cultural, and strategic challenges that threaten its performance and competitive position. The immediate trigger for organizational strain is the significantly delayed release of Sankofa Insights 4.0, an advanced analytics suite designed for high-profile public sector agencies.

\noindent Staff feedback reveals widespread concerns: engineers hesitate to experiment due to fear of reprimand, data scientists express burnout from unrealistic expectations, designers feel excluded from decision-making processes, and most staff lack understanding of the overarching vision. These symptoms point to fundamental leadership and organizational climate issues rather than technical capacity problems.

\noindent This report applies Goleman's (2000) framework to diagnose leadership styles, assess their impact on organizational climate drivers, identify emotional intelligence gaps, and propose evidence-based interventions to restore organizational health and deliver Insights 4.0 successfully.

\section{Leadership Diagnosis: Current Styles at SDS}

\noindent Our analysis identifies four dominant leadership styles currently operating at SDS, each contributing to the organizational challenges in distinct ways.

\subsection{Coercive Leadership Style: CEO}

\noindent The CEO frequently employs a coercive, deadline-centered approach intended to demand fast results. Multiple pieces of evidence from the case study support this diagnosis:

\begin{itemize}
    \item Engineers consistently report hesitating to experiment out of fear of reprimand, indicating a climate of fear that suppresses innovation.
    \item Data scientists express burnout resulting from unrealistic expectations and constant reprioritization.
    \item The deadline-centered approach and demand for fast results prevent the design team from feeling heard or valued in the process.
    \item Most staff members lack understanding of the overarching vision for Insights 4.0, creating a paradox where high pressure exists without strategic direction.
\end{itemize}

\noindent This coercive approach creates a climate where employees shift from proactive decision-making to simply avoiding errors, resulting in bottlenecks and disintegrated execution.

\subsection{Pacesetting and Micromanagement: CTO}

\noindent The CTO exhibits a pacesetting leadership style combined with micromanagement, characterized by high performance expectations and a tendency to take over technical work when others do not meet his pace. Evidence from the case study reveals several key indicators:

\begin{itemize}
    \item Data scientists express burnout resulting from unrealistic expectations and constant reprioritization.
    \item The CTO alternates between pacesetting behaviors, taking on technical work himself to maintain speed, and micromanagement that undermines team autonomy.
    \item Staff morale has deteriorated significantly under this leadership approach.
    \item The absence of mentoring and coaching despite evolving technical requirements represents a critical failure. Pacesetting emphasizes self-direction, but employees are not provided with coaching or mentoring, with an assumption that they will complete the work independently.
    \item This micromanagement approach makes designers feel excluded from decision-making processes, further fragmenting team cohesion.
\end{itemize}

\subsection{Democratic Leadership Style: HR Director}

\noindent The HR Director relies heavily on democratic decision-making processes, which, while inclusive in principle, leads to prolonged discussions and slow execution. The case study provides the following evidence:

\begin{itemize}
    \item The HR Director relies heavily on democratic decision-making, leading to prolonged discussions and slow execution.
    \item The slowness of execution and prolonged discussions directly lead to delays in task completion and decision-making, which represents a key drawback of an overused democratic style.
    \item When discussions become endless with no clear direction, employees may feel leaderless and confused, which undermines their sense of responsibility and contributes to low morale.
\end{itemize}

\noindent The democratic style generally fosters flexibility by giving workers a voice in decision-making. However, prolonged discussions create bureaucratic hurdles that hinder action and slow down innovation.

\subsection{Affiliative Leadership Style: Innovation Lead}

\noindent The Innovation Lead adopts a largely affiliative style, prioritizing harmony but avoiding difficult conversations and critical feedback. Evidence from the case study demonstrates the following:

\begin{itemize}
    \item The Innovation Lead prioritizes harmony but avoids difficult conversations and critical feedback.
    \item Despite the affiliative nature of this leadership approach, significant unresolved tension persists within development teams.
    \item Followers lack a clear sense of direction due to the absence of constructive feedback and coordination.
    \item While the affiliative style offers ample positive feedback, failure to address poor performance allows unresolved tension to persist, causing negative impact on commitment and deeper morale issues.
\end{itemize}

\noindent This leadership style has minimal impact on flexibility, allowing employees room to work without unnecessary strictures. However, it negatively impacts responsibility because it fosters avoidance of corrective feedback and difficult conversations. Standards suffer because performance and work go unsupervised and uncorrected, causing delays and poor output.

\section{Missing Leadership Styles}

\noindent Our analysis reveals two critical leadership styles that are absent from SDS's leadership repertoire, both of which are essential for addressing current organizational challenges.

\subsection{Authoritative (Visionary) Leadership}

\noindent Evidence from the case study indicates that SDS lacks authoritative leadership. Engineers consistently cite inconsistent direction and shifting priorities as major obstacles. The current leadership vacuum leaves teams without a clear North Star or stable roadmap for Insights 4.0.

\noindent An authoritative leadership style would provide clear direction and alignment across teams, converting pressure into purpose and stabilizing priorities. The authoritative style increases clarity and commitment during final development phases while allowing flexibility in execution. This approach would address the fundamental problem of strategic ambiguity that currently plagues the organization.

\subsection{Coaching Leadership}

\noindent Evidence from the case study reveals that coaching leadership is absent at SDS. Staff consistently report that mentoring and coaching are absent despite evolving technical requirements. Engineers hesitate to experiment, limiting innovation and ownership.

\noindent A coaching leadership style would build capacity amid evolving requirements, restore mentoring and one-on-one support, improve autonomy, reduce burnout, and create safe spaces for experimentation and growth. Coaching fosters employees' belief in their capacity to continuously rise to challenges, which is essential for a technology company facing evolving technical requirements.

\section{Impact on Organizational Climate}

\noindent Goleman (2000) identifies six key climate drivers that determine organizational performance: flexibility, responsibility, standards, rewards, clarity, and commitment. Our analysis reveals how current leadership styles negatively impact each of these drivers.

\subsection{Impact of Coercive Leadership}

The CEO's coercive style creates a climate of fear and pressure throughout the organization. Engineers hesitate to experiment out of fear of reprimand, illustrating the negative impact on flexibility. When employees fear serious consequences for mistakes, they avoid taking initiative, suppressing creativity and slowing problem-solving.

Responsibility weakens as staff shift from proactive decision-making to simply avoiding errors. Employees are unlikely to take ownership of their work due to fear of reprimand, resulting in bottlenecks and disintegrated execution.

The coercive focus on speed creates distorted standards because employees perceive expectations as punitive and unrealistic. This leads to anxiety and contributes to declining morale. Rewards are absent in this climate, as emphasis is on compliance rather than recognition, discouraging discretionary effort and reducing motivation.

Clarity is severely weakened. Although coercive leaders demand performance, most staff lack understanding of the overarching vision, highlighting the paradox of high pressure without strategic direction. This absence of clarity creates misalignment, causing rework and further delays.

Finally, coercive leadership deteriorates commitment due to the collapse of psychological safety. Fear-driven climates lead to employees' emotional disengagement, resulting in lower morale and reduced productivity. The CEO's coercive style negatively affects all climate drivers.

\subsection{Impact of Pacesetting and Micromanagement}

The CTO's combination of pacesetting and micromanagement significantly damages organizational climate. Flexibility suffers because the CTO constantly shifts priorities and steps in to do work himself, preventing employees from exploring solutions independently. Constant reprioritization overwhelms staff and makes adaptation difficult, slowing innovation and contributing to continuous delays.

Responsibility sharply reduces because when leaders do tasks that employees should perform, employees lose autonomy and initiative. The case reading explicitly states that the CTO often takes on technical work himself to maintain speed, which is a classic indicator of pacesetting that reduces responsibility.

Clarity suffers because, although the intention is to move fast, there is always confusion about priorities due to rapid changes in direction. The combination of pacesetting and micromanagement creates situations where employees receive moment-to-moment instructions but never get stable direction. This is reflected in inconsistent expectations and unclear workflows, leading to misalignment and avoidable delays.

Standards become counterproductive when perceived as unrealistic, especially by data scientists experiencing high pressure and fatigue. Such standards become counterproductive when employees lack support, ultimately leading to errors, rework, and delays.

Rewards are minimal under this style. Pacesetters often assume competence and rarely provide positive reinforcement. Without recognition, teams feel undervalued, contributing to declining morale. Commitment declines because people lose confidence in the process. Burnout, pressure, and lack of autonomy make employees disengage emotionally, often considering employment elsewhere.

\subsection{Impact of Democratic Leadership}

The HR Director's democratic style has mixed impacts on climate drivers. Flexibility initially benefits from democratic decision-making, but prolonged discussions create bureaucratic hurdles that hinder action, slowing down actual innovation.

Responsibility initially increases because democracy gives employees a sense of organizational responsibility. They feel responsible to help the organization grow because they can contribute their voice in decision-making. However, if discussions become endless with no clear direction, employees may feel leaderless and confused, undermining their sense of responsibility and contributing to low morale.

Standards have mixed impact. The style enables realistic goal-setting as individuals have a say in setting their own standards. However, the focus on consensus might dilute challenging goals, potentially leading to mediocre standards.

Commitment initially is high because employees appreciate being included, but it drops sharply when prolonged discussions fail to produce action. Clarity is the most negatively impacted dimension. In trying to reach consensus, roles become unclear, decision rights become ambiguous, and next steps remain undefined.

\subsection{Impact of Affiliative Leadership}

The Innovation Lead's affiliative style has minimal impact on flexibility, allowing employees room to work without unnecessary strictures. However, responsibility suffers because this leadership style fosters avoidance of corrective feedback and difficult conversations.

Standards are negatively impacted because this style allows performance and work to go unsupervised and uncorrected, causing delays and poor output. Rewards offer ample positive feedback, but this may negatively affect morale as failure to address poor performance allows unresolved tension to persist.

Clarity has negative impact because followers lack a clear sense of direction due to the absence of constructive feedback and coordination. Though this style should foster belonging and loyalty, the unresolved tension indicates otherwise, causing negative impact on commitment and deeper morale issues.

\section{Organizational Outcomes and Climate Deterioration}

\noindent The deterioration of climate drivers directly manifests in three critical organizational problems: delays in Insights 4.0, morale issues, and poor coordination.

\subsection{Delays in Insights 4.0}

The primary contributing factors are low flexibility resulting from prolonged meetings and low standards from uncorrected performance. The main contributing leadership failures are the overuse of democratic style and affiliative style. Evidence includes engineers hesitating to experiment and constant reprioritization, both of which contribute directly to project delays.

\subsection{Morale Issues}

Contributing factors include low rewards, where there is a gap between praise and reality, along with low responsibility and low commitment. The main contributing leadership failure is the overuse of affiliative style. Evidence includes data scientists expressing burnout and staff morale having deteriorated significantly.

\subsection{Poor Coordination}

Contributing factors include low clarity, where confusion exists with no clear directives, and low responsibility, where there is no corrective alignment. The main contributing leadership failures are the overuse of both democratic and affiliative styles. Evidence includes coordination between departments having weakened and unclear roles and responsibilities across departments.

\section{Emotional Intelligence Deficits}

\noindent Our analysis reveals critical emotional intelligence deficits among the executive leadership team that contribute significantly to the organizational challenges. Four key deficits are identified: self-awareness, self-management, social awareness, and social skills.

\subsection{Self-Awareness Deficiency}

\noindent Leaders at SDS demonstrate a critical lack of self-awareness regarding how their leadership behaviors impact the organization. The CEO appears unaware that his coercive, deadline-centered approach creates a climate of fear that suppresses innovation and reduces psychological safety. The CTO does not recognize that his pacesetting and micromanagement behaviors undermine team autonomy and contribute to burnout. The HR Director seems unaware that prolonged democratic discussions are causing delays and confusion. The Innovation Lead fails to recognize that avoiding difficult conversations is allowing unresolved tension to persist, despite the affiliative style's intention to create harmony.

\noindent This deficit in self-awareness prevents leaders from understanding their own emotional triggers, recognizing how their actions affect others, and adjusting their leadership styles appropriately. Without self-awareness, leaders cannot effectively develop other emotional intelligence competencies or adapt their leadership approaches to different situations.

\subsection{Self-Management Needs}

\noindent The CTO reacts with micromanagement while the CEO reacts with coercion. Both leaders require development in strategic response versus reaction, emotional control, and stress management. This deficit creates reactive, inconsistent leadership that undermines stability and prevents teams from developing effective working patterns.

\subsection{Social Awareness Required}

\noindent Leaders appear disconnected from staff needs and lack empathy for burnout and concerns. They require development in active listening, perspective-taking, and understanding staff emotional states. This disconnect between leadership behaviors and staff needs creates a cycle of deteriorating morale and prevents building relationships necessary for collaboration.

\subsection{Social Skills Deficiency}

\noindent The HR Director and Innovation Lead avoid conflicts and demonstrate poor communication and coordination capabilities. They require development in conflict management capabilities, clear messaging, and effective facilitation. This deficit prevents effective leadership style adaptation and coordination, exacerbating organizational problems.

\section{Leadership Intervention Plan}

\noindent Leadership intervention at Sankofa Digital Solutions is both critical and urgent. Without intervention, SDS risks not only a failed launch but also complete organizational failure. Detailed below is a comprehensive plan for immediate and long-term intervention, inspired by Goleman's (2000) leadership framework.

\subsection{CEO: Transformation to Authoritative and Affiliative Styles}

The CEO should shift away from the coercive, deadline-centered, fast-results-oriented leadership style. Instead, a more authoritative yet open and affiliative approach would yield better results. Emotional intelligence development, specifically social awareness and social skills, is essential here.

He should acknowledge the difficulty of the company's current situation and the effort everyone has invested in the product so far. He can then re-articulate the vision for SDS's aspirations, allowing his executive team to provide feedback on how he can better support them in reaching this milestone. This approach would de-escalate organizational pressure and create space for the executive team to communicate candidly about the company's capacity and potential timeline adjustments.

Moreover, this would empower the CTO, Innovation Lead, and HR Manager to address directly the concerns of Engineers, Designers, and Data Scientists. The CEO's leadership style may also be a response to the board's insistence on execution speed and the importance of the current customer. Therefore, to create slack time for SDS, he can reframe the change in company strategy in terms of shared interests such as long-term shareholder value or improved customer satisfaction.

\subsection{CTO: Transformation to Authoritative and Coaching Styles}

For the CTO, it is evident that he is under pressure to deliver and has lost sight of the bigger picture. Consequently, he pacesets by taking on work himself and micromanages. These tendencies limit team members' flexibility to try, experiment, and innovate.

He must adopt a more authoritative and coaching leadership approach: authoritative to ensure consistent direction across Design, Data Science, and Engineering teams, and coaching to cultivate a growth mindset within them. Self-management and social awareness are especially critical here. Pacesetting may be a self-soothing way of coping with frustration over others' weaknesses; however, it is ultimately inefficient, suppresses curiosity and autonomy, and only heightens employees' awareness of their inadequacies.

Coaching, on the other hand, fosters employees' belief in their capacity to continuously rise to challenges. The HR Director and Innovation Lead could be particularly helpful in supporting this shift.

\subsection{HR Director and Innovation Lead: Strengthening Social Skills}

The HR Director and Innovation Lead currently use leadership styles positively associated with a healthy company climate: democratic decision-making and affiliative leadership, respectively. However, these approaches remain insufficient when used independently or as mechanisms to avoid difficult conversations. Both lack the social-skill component of emotional intelligence.

Rather than avoiding conflict, they should strengthen their communication and conflict-management capabilities. They should become proficient in listening, delivering clear and well-tailored messages, resolving disagreements, and facilitating solutions effectively. Clear messaging and direction enable employees to act more decisively because they have a firm sense of purpose and a clear understanding of their role in the company.

\section{Ninety-Day Action Plan}

\noindent In the next 90 days, the following actions represent a comprehensive starting point for organizational transformation.

\subsection{Immediate Actions: Days 1-30}

The first phase requires emergency interventions to stabilize the organization. The CEO should meet with the executive team, acknowledge the pressures within the company, and solicit feedback on the current status and possible short-term mitigation strategies. This emergency leadership team meeting should create space for candid dialogue.

An emergency meeting between the executive team and the Board should discuss company pressures and potential timeline adjustments. This requires careful preparation to reframe the situation in terms of long-term value rather than short-term failure.

Customer engagement should explore possible timeline adjustments, ensuring that client relationships remain strong while creating realistic expectations. Finally, a company-wide communication from the CEO should acknowledge current pressures, outline short-term adjustments, and announce a long-term company-climate evaluation allowing all employees to participate meaningfully.

\subsection{Short-Term Actions: Days 31-60}

The second phase focuses on assessment and structural improvements. An internal company-climate audit should be led by the HR Director and co-led by the CEO. This audit should measure all six climate drivers using Goleman's framework and provide baseline data for improvement tracking.

Internal structural adjustments for the Engineering, Data Science, and Design teams should be co-led by the CTO, Innovation Lead, and HR Director. These adjustments should clarify roles, improve coordination, and establish clear decision-making processes.

Emotional intelligence development workshops should focus on self-awareness, self-management, social awareness, and social skills for all executive team members. These workshops should be practical, providing tools and techniques that leaders can immediately apply.

\subsection{Mid-Term Actions: Days 61-90}

The third phase focuses on reinforcement and measurement. New leadership behaviors should be reinforced through regular check-ins and coaching sessions. Climate improvements should be measured using Goleman's climate drivers, comparing results to the baseline established in the audit phase.

Progress and early wins should be celebrated to build momentum. This includes recognizing improvements in coordination, reductions in delays, and improvements in staff morale. These celebrations should be public and should acknowledge both individual and team contributions.

\section{Expected Outcomes}

\noindent By Day 90, we expect to see significant improvements across all six climate drivers. Flexibility should increase as fear decreases and experimentation is encouraged. Responsibility should improve as leaders step back from micromanagement and employees gain autonomy. Standards should become more realistic and achievable while maintaining high quality expectations.

Rewards should become more meaningful and connected to actual performance. Clarity should improve dramatically as authoritative leadership provides clear direction and democratic processes are used appropriately. Commitment should increase as psychological safety is restored and employees feel valued and heard.

These climate improvements should manifest in accelerated progress on Insights 4.0 development milestones, enhanced staff morale and commitment through psychological safety restoration, better coordination and collaboration between departments, and sustainable leadership practices with emotional intelligence development integration.

\section{Limitations and Future Considerations}

\noindent It is important to note that the hypothesis that SDS's outcomes are primarily hindered by organizational climate, which is itself hindered by leadership behaviors driven by weak application of emotional intelligence, represents one explanatory theory. Other factors may also influence SDS's outcomes, climate, or leadership styles, such as the fast-paced demands of their market, the nature of their customers including government and public institutions, or even their recruitment pool.

To understand these other variables, a more systemic approach such as the Iceberg Model, which goes beyond emotional intelligence to examine the mental models and structures that influence all stakeholders, would be appropriate. Nevertheless, in the short to mid-term, most of SDS's challenges are addressable by improving leadership styles and emotional intelligence competencies.

\section{Conclusion}

\noindent This analysis reveals that Sankofa Digital Solutions faces significant challenges stemming from inappropriate leadership styles and emotional intelligence deficits. The coercive, pacesetting, overused democratic, and conflict-avoidant affiliative styles currently in use are creating a climate that undermines flexibility, responsibility, standards, rewards, clarity, and commitment.

The introduction of authoritative and coaching leadership styles, combined with strengthened social skills across the executive team, provides a clear path forward. The 90-day action plan offers a structured approach to transformation that addresses immediate crises while building long-term organizational health.

Success requires executive commitment to emotional intelligence development and consistent application of new leadership styles. With this commitment, SDS can restore organizational climate, deliver Insights 4.0 successfully, and rebuild its reputation for technical excellence and organizational agility.

\section{References}

\noindent Goleman, D. (2017). Leadership that gets results. Leadership Academy NHS. \url{https://content.leadershipacademy.nhs.uk/aspce3/files/Leadership\_that\_gets\_results\_goleman.pdf}

\noindent Sterne, B. (2012, November 22). A review of Leadership that gets results by Daniel Goleman. \url{https://brendansterne.com/2012/11/22/a-review-of-leadership-that-gets-results-by-daniel/}

\noindent Harvard Business Review. (2025, October). 5 critical skills leaders need in the age of AI. \url{https://hbr.org/2025/10/5-critical-skills-leaders-need-in-the-age-of-ai}

\section{Appendix: Presentation Link}

\noindent A detailed visual presentation of our analysis is available online at:

\noindent \url{https://raw.githack.com/Thomas-A1/Leadership\_With\_Results/main/Leadership\_Presentation.html}

\noindent This presentation provides a comprehensive overview of our findings, including leadership diagnosis, climate impact analysis, emotional intelligence gaps, and the recommended intervention plan.

\end{document}

